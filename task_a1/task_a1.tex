\documentclass{article}

\usepackage{fancyhdr}
\usepackage{extramarks}
\usepackage{amsmath}
\usepackage{amssymb}
\usepackage{amsthm}
\usepackage{amsfonts}
\usepackage{tikz}
\usepackage[plain]{algorithm}
\usepackage{algpseudocode}
\usepackage{graphicx}
\usepackage{forest}
\usepackage{pdfpages}
\usepackage{catchfile}
\usepackage{hyperref}

\topmargin=-0.45in
\evensidemargin=0in
\oddsidemargin=0in
\textwidth=6.5in
\textheight=9.0in
\headsep=0.25in


\linespread{1.1}

\pagestyle{fancy}

\lhead{Joel Ho Eng Kiat A0200385N}
\chead{CS3219}
\rhead{Task A1}
\lfoot{\lastxmark}
\cfoot{\thepage}

\begin{document}
    Link to GitHub repo \href{https://github.com/JoelHo/cs3219-assignments/tree/master/task_a1}{https://github.com/JoelHo/cs3219-assignments/tree/master/task\_a1}\\

    To run the Docker containers:

    \texttt{docker-compose build}

    \texttt{docker-compose up -d}\\

    Then go to \href{http://localhost:8080/}{http://localhost:8080/} to view the default page served on the reverse proxy. It contains links to access the different servers hidden behind it, which are networked to the proxy via docker.

    This can be verified by looking at the server IP and source url that \texttt{nginx} will replace in the response on each server.\\

    This is an implementation of a reverse proxy using \texttt{nginx}, forwarding requests to 2 different webservers (also on \texttt{nginx}). This was chosen to demonstrate better the one of the typical use cases of a reverse proxy, to forward requests to 2 different servers.
    This can be verified by looking at the server ip and source url that nginx will replace in the response on each server.\\

    While I have setup a simple reverse proxy on nginx before for my personal projects, using Docker was relatively new to me, and thus I wanted to learn how to setup a multi-container Docker application (since that is one of the primary applications of Docker). In doing so, I have learnt more amount about using \texttt{docker-compose}, and how Docker networking works, through trial and error.
\end{document}
